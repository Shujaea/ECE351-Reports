%%%%%%%%%%%%%%%%%%%%%%%%%%%%%%%%%%%%%%%%%%%
%%% DOCUMENT PREAMBLE %%%
\documentclass[12pt]{report}
\usepackage[english]{babel}
%\usepackage{natbib}
\usepackage{url}
\usepackage[utf8x]{inputenc}
\usepackage{amsmath}
\usepackage{graphicx}
\graphicspath{{images/}}
\usepackage{parskip}
\usepackage{fancyhdr}
\usepackage{vmargin}
\usepackage{listings}
\usepackage{hyperref}
\usepackage{xcolor}

\definecolor{codegreen}{rgb}{0,0.6,0}
\definecolor{codegray}{rgb}{0.5,0.5,0.5}
\definecolor{codeblue}{rgb}{0,0,0.95}
\definecolor{backcolour}{rgb}{0.95,0.95,0.92}

\lstdefinestyle{mystyle}{
    backgroundcolor=\color{backcolour},   
    commentstyle=\color{codegreen},
    keywordstyle=\color{codeblue},
    numberstyle=\tiny\color{codegray},
    stringstyle=\color{codegreen},
    basicstyle=\ttfamily\footnotesize,
    breakatwhitespace=false,         
    breaklines=true,                 
    captionpos=b,                    
    keepspaces=true,                 
    numbers=left,                    
    numbersep=5pt,                  
    showspaces=false,                
    showstringspaces=false,
    showtabs=false,                  
    tabsize=2
}
 
\lstset{style=mystyle}

\setmarginsrb{3 cm}{2.5 cm}{3 cm}{2.5 cm}{1 cm}{1.5 cm}{1 cm}{1.5 cm}

\title{Lab 10}								
% Title
\author{ Shujaea Aldousari}						
% Author
\date{Nov 10, 2021}
% Date

\makeatletter
\let\thetitle\@title
\let\theauthor\@author
\let\thedate\@date
\makeatother

\pagestyle{fancy}
\fancyhf{}
\rhead{\theauthor}
\lhead{\thetitle}
\cfoot{\thepage}
%%%%%%%%%%%%%%%%%%%%%%%%%%%%%%%%%%%%%%%%%%%%
\begin{document}

%%%%%%%%%%%%%%%%%%%%%%%%%%%%%%%%%%%%%%%%%%%%%%%%%%%%%%%%%%%%%%%%%%%%%%%%%%%%%%%%%%%%%%%%%

\begin{titlepage}
	\centering
    \vspace*{0.5 cm}
   % \includegraphics[scale = 0.075]{bsulogo.png}\\[1.0 cm]	% University Logo
\begin{center}    \textsc{\Large   ECE 351 - Section 52 \# }\\[2.0 cm]	\end{center}% University of Idaho
	\textsc{\Large Lab 10 Report  }\\[0.5 cm]				% Course Code
	\rule{\linewidth}{0.2 mm} \\[0.4 cm]
	{ \huge \bfseries \thetitle}\\
	\rule{\linewidth}{0.2 mm} \\[1.5 cm]
	
	\begin{minipage}{0.4\textwidth}
		\begin{flushleft} \large
		%	\emph{Submitted To:}\\
		%	Name\\
          % Affiliation\\
           %contact info\\
			\end{flushleft}
			\end{minipage}~
			\begin{minipage}{0.4\textwidth}
            
			\begin{flushright} \large
			\emph{Submitted By :} \\
			Shujaea Aldousari  
		\end{flushright}
           
	\end{minipage}\\[2 cm]
	
%	\includegraphics[scale = 0.5]{PICMathLogo.png}
    
    
    
    
	
\end{titlepage}

%%%%%%%%%%%%%%%%%%%%%%%%%%%%%%%%%%%%%%%%%%%%%%%%%%%%%%%%%%%%%%%%%%%%%%%%%%%%%%%%%%%%%%%%%

\tableofcontents
\pagebreak

%%%%%%%%%%%%%%%%%%%%%%%%%%%%%%%%%%%%%%%%%%%%%%%%%%%%%%%%%%%%%%%%%%%%%%%%%%%%%%%%%%%%%%%%%
\renewcommand{\thesection}{\arabic{section}}
\section{Introduction}

In this lab, we were working on Frequency response.  The main idea of this lab is to be familiar with frequency response tools and Bode plots using Python.  Before the lab, we were working on the pre-lab which we used the magnitude formula and phase formula As shown below 


\\
$$H(s) = \frac{\frac{s}{RC}}{s^2+\frac{s}{RC}+\frac{1}{LC}}$$
$$|H(j\omega)| = \frac{\frac{\omega}{RC}}{\sqrt{(\frac{1}{LC}-\omega^2)^2+(\frac{\omega}{RC})^2}}$$
$$\angle H(j\omega) = 90- \ tan^-1(\frac{\frac{\omega}{RC}}{\frac{1}{LC}-\omega^2})$$






\section{Part 1}

In part one, we were asked to get and plot the equations that we got from the pre-lab and use Python in order to develop it. In the next task, we were asked to use scipy.signal.bode to plot the magnitude and phase frequency response for the RLC transfer function.  After that, we got the plots out of the bode plot and compared them with the built-in function. In the last task, we were working with the Control Package. The example code was given in the lab Manuel which helped us to plot the bode plot by using the frequency domain.

    






\includegraphics[scale=0.5]{Figure 1.png} 
\includegraphics[scale=0.5]{Figure 2.png} 
\includegraphics[scale=0.5]{Figure 3.png} 
  caption{Caption}

  end{figure}


\section{part 2}

In the last part of this lab, we were asked to Use the frequency response model developed in Part 1 as a filter for a multi-band input signal. First, we had to plot the given x(t) function. After that We were asked to use the Z-domain for the transfer function H(s) that we got by using scipy.signal.bilinear(). Finally, we used scipy.signal.lfilter() to pass the input signal x(t) through the filter, after that we plotted y(t).

\includegraphics[scale=0.5]{Figure 4.png} 
\includegraphics[scale=0.5]{Figure 5.png} 


\section{Questions}

1. Explain how the filter and filtered output in Part 2 makes sense given the Bode plots from
Part 1. Discuss how the filter modifies specific frequency bands, in Hz.

According to bode plot in task 3 part 1, we can see that the center of the frequency is around 3000 Hz and the input signal has 3 frequencies which they were 100, 3024 and 50,000 Hz, We can see that the frequency at 100 Hz is much smaller than 100 Hz as well as the 50,000 Hz is much bigger than 3000 Hz  So they won't pass expect for the 3024 Hz because it is very close to the center frequency.

2. Discuss the purpose and workings of
scipy.signal.bilinear() and scipy.signal.lfilter().

The use of scipy.signal.bilinear() is to conver the transfer function into z-domain

The use of scipy.signal.lfilter() is to pass the input signal through the filter

3. What happens if you use a different sampling frequency in scipy.signal.bilinear() than
you used for the time-domain signal?

The shape of x(t) will be different, also the use of the filter will be different.

4. Leave any feedback on the clarity of lab tasks, expectations, and deliverables.

Overall, the lab was clear.






\section{conclusion}

In this lab, we deal with RLC circuit and we deal with frequency response. In the beginning, we used Magnitude formula as well as the phase formula from the pre- lab and plotted the Magnitude and the phase with radian frequency. Then, we used the built in function signal.bode to plot the Magnitude and the phase. After that, we used control function to bode plot the frequency response of the transfer function with respect to Hz. In the second part, we deal with input signal x(t), we plotted x(t) the input signal. Then, we used signal.bilinear to convert the transfer function to z-domain. Finally, we used signal.lfilter to pass the input signal through the filter to get the output. At the end, we plotted output signal y(t) with respect to time. The lab focuses on the frequency response and how we can deal with the filter and bode plot.




\newpage



\end{document}

%This template was created by Roza Aceska.