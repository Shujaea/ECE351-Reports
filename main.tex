%%%%%%%%%%%%%%%%%%%%%%%%%%%%%%%%%%%%%%%%%%%
%%% DOCUMENT PREAMBLE %%%
\documentclass[12pt]{report}
\usepackage[english]{babel}
%\usepackage{natbib}
\usepackage{url}
\usepackage[utf8x]{inputenc}
\usepackage{amsmath}
\usepackage{graphicx}
\graphicspath{{images/}}
\usepackage{parskip}
\usepackage{fancyhdr}
\usepackage{vmargin}
\usepackage{listings}
\usepackage{hyperref}
\usepackage{xcolor}

\definecolor{codegreen}{rgb}{0,0.6,0}
\definecolor{codegray}{rgb}{0.5,0.5,0.5}
\definecolor{codeblue}{rgb}{0,0,0.95}
\definecolor{backcolour}{rgb}{0.95,0.95,0.92}

\lstdefinestyle{mystyle}{
    backgroundcolor=\color{backcolour},   
    commentstyle=\color{codegreen},
    keywordstyle=\color{codeblue},
    numberstyle=\tiny\color{codegray},
    stringstyle=\color{codegreen},
    basicstyle=\ttfamily\footnotesize,
    breakatwhitespace=false,         
    breaklines=true,                 
    captionpos=b,                    
    keepspaces=true,                 
    numbers=left,                    
    numbersep=5pt,                  
    showspaces=false,                
    showstringspaces=false,
    showtabs=false,                  
    tabsize=2
}
 
\lstset{style=mystyle}

\setmarginsrb{3 cm}{2.5 cm}{3 cm}{2.5 cm}{1 cm}{1.5 cm}{1 cm}{1.5 cm}

\title{Lab5}								
% Title
\author{ Shujaea Aldousari}						
% Author
\date{Oct 6, 2021}
% Date

\makeatletter
\let\thetitle\@title
\let\theauthor\@author
\let\thedate\@date
\makeatother

\pagestyle{fancy}
\fancyhf{}
\rhead{\theauthor}
\lhead{\thetitle}
\cfoot{\thepage}
%%%%%%%%%%%%%%%%%%%%%%%%%%%%%%%%%%%%%%%%%%%%
\begin{document}

%%%%%%%%%%%%%%%%%%%%%%%%%%%%%%%%%%%%%%%%%%%%%%%%%%%%%%%%%%%%%%%%%%%%%%%%%%%%%%%%%%%%%%%%%

\begin{titlepage}
	\centering
    \vspace*{0.5 cm}
   % \includegraphics[scale = 0.075]{bsulogo.png}\\[1.0 cm]	% University Logo
\begin{center}    \textsc{\Large   ECE 351 - Section 52 \# }\\[2.0 cm]	\end{center}% University of Idaho
	\textsc{\Large Lab 5 Report  }\\[0.5 cm]				% Course Code
	\rule{\linewidth}{0.2 mm} \\[0.4 cm]
	{ \huge \bfseries \thetitle}\\
	\rule{\linewidth}{0.2 mm} \\[1.5 cm]
	
	\begin{minipage}{0.4\textwidth}
		\begin{flushleft} \large
		%	\emph{Submitted To:}\\
		%	Name\\
          % Affiliation\\
           %contact info\\
			\end{flushleft}
			\end{minipage}~
			\begin{minipage}{0.4\textwidth}
            
			\begin{flushright} \large
			\emph{Submitted By :} \\
			Shujaea Aldousari  
		\end{flushright}
           
	\end{minipage}\\[2 cm]
	
%	\includegraphics[scale = 0.5]{PICMathLogo.png}
    
    
    
    
	
\end{titlepage}

%%%%%%%%%%%%%%%%%%%%%%%%%%%%%%%%%%%%%%%%%%%%%%%%%%%%%%%%%%%%%%%%%%%%%%%%%%%%%%%%%%%%%%%%%

\tableofcontents
\pagebreak

%%%%%%%%%%%%%%%%%%%%%%%%%%%%%%%%%%%%%%%%%%%%%%%%%%%%%%%%%%%%%%%%%%%%%%%%%%%%%%%%%%%%%%%%%
\renewcommand{\thesection}{\arabic{section}}
\section{Introduction}
 

In this lab, we worked on Step and Impulse Response of a RLC Band Pass Filter and we learned in how to use Laplace transform, Since we were given the function that’s on the pre-lab, it helped us to find the impulse response. In the pre-lab, It is clear that I used a cosin function which has not been used, however I used the function H(s) that was given.




\section{Part 1}

By using the pre-lab function that were given, we were able to create our user defined function in order to get the impulse response and built in functions. They both have the same results as showing in below.





\begin{lstlisting}[language=Python]




import numpy as np
import matplotlib . pyplot as plt
import scipy . signal as sig

steps = 1e-6 # Define step size
t = np. arange (0, 1.2e-3 + steps , steps ) 
R = 1e3
L= 27e-3
C= 100e-9

# user defined function
def step (t): 
    y = np. zeros (t. shape ) 
    for i in range ( len (t)): 
        if t[i] >= 0:  
            y[i] = 1
        else:
            y[i] = 0
    return y 

def h(R, L, C, t):
    Alpha = -1 / (2 * R * C)
    w = 0.5 * np.sqrt(((1/(R*C))**2) - 4/(L*C) + 0*1j)
    P = Alpha + w 
    g = P / (R*C)
    gMag = np.abs(g)
    gRad = np.angle(g)
    y = (gMag / np.abs(w)) * np.exp(Alpha*t) * np.sin(np.abs(w) * t + gRad) * step (t) 
    return y

num = [0, 1e4 , 0]
den = [1, 1e4 , 370.4e6]
tout, hout = sig. impulse (( num , den), T = t)

plt. figure ( figsize = (10 , 7))
plt. subplot (2, 1, 1)
plt. plot (t, h(R, L, C, t))
plt. grid ()
plt. ylabel ('h1(t)')
plt. title (' Impulse responce')
plt. subplot (2, 1, 2)
plt. plot (tout, hout)
plt. grid ()
plt. ylabel ('h2(t)')
plt. xlabel ('t')

plt. show ()





\end{lstlisting}

\includegraphics{1.png}

\section{Part 2 }

In part 2, we had to get the step response of the circuit then, we compute the final value theorem of the step response after that we compare this results with that found in part 1. 

\begin{lstlisting}[language=Python]
import numpy as np
import matplotlib . pyplot as plt
import scipy . signal as sig

steps = 1e-6 # Define step size
t = np. arange (0, 1.2e-3 + steps , steps ) 


num = [0, 1e4 , 0]
den = [1, 1e4 , 370.4e6]
tout, yout = sig. step (( num , den), T = t)

plt. figure ( figsize = (10 , 7))
plt. plot (tout, yout)
plt. grid ()
plt. title (' Step responce')
plt. ylabel ('y(t)')
plt. xlabel ('t')

plt. show ()


\end{lstlisting}

\includegraphics{2.png}



\section{Questions}
Q1: Explain the result of the Final Value Theorem from Part 2 Task 2 in terms of the physical
circuit components.
At t= infinity, the capacitor will be open circuit and the inductor will act short circuit. Then the voltage Vout is parallel with the inductor which is equal to zero. which agrees with the final value theorem which is equal to zero.  





\section{conclusion}


The main purpose of this lab is how to use and be familiar with Laplace transform. By creating the user defined for the impulse response after that we check the result with the result that are generated by using the impulse built in function. The step response of the circuit is generated by using step built in function then the final value thermos of the step response was calculated and the result agreed with the result of part 1.

\newpage



\end{document}

%This template was created by Roza Aceska.