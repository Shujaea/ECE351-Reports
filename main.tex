%%%%%%%%%%%%%%%%%%%%%%%%%%%%%%%%%%%%%%%%%%%
%%% DOCUMENT PREAMBLE %%%
\documentclass[12pt]{report}
\usepackage[english]{babel}
%\usepackage{natbib}
\usepackage{url}
\usepackage[utf8x]{inputenc}
\usepackage{amsmath}
\usepackage{graphicx}
\graphicspath{{images/}}
\usepackage{parskip}
\usepackage{fancyhdr}
\usepackage{vmargin}
\usepackage{listings}
\usepackage{hyperref}
\usepackage{xcolor}

\definecolor{codegreen}{rgb}{0,0.6,0}
\definecolor{codegray}{rgb}{0.5,0.5,0.5}
\definecolor{codeblue}{rgb}{0,0,0.95}
\definecolor{backcolour}{rgb}{0.95,0.95,0.92}

\lstdefinestyle{mystyle}{
    backgroundcolor=\color{backcolour},   
    commentstyle=\color{codegreen},
    keywordstyle=\color{codeblue},
    numberstyle=\tiny\color{codegray},
    stringstyle=\color{codegreen},
    basicstyle=\ttfamily\footnotesize,
    breakatwhitespace=false,         
    breaklines=true,                 
    captionpos=b,                    
    keepspaces=true,                 
    numbers=left,                    
    numbersep=5pt,                  
    showspaces=false,                
    showstringspaces=false,
    showtabs=false,                  
    tabsize=2
}
 
\lstset{style=mystyle}

\setmarginsrb{3 cm}{2.5 cm}{3 cm}{2.5 cm}{1 cm}{1.5 cm}{1 cm}{1.5 cm}

\title{Final Project}								
% Title
\author{ Shujaea Aldousari}						
% Author
\date{Dec, 9, 2021}
% Date

\makeatletter
\let\thetitle\@title
\let\theauthor\@author
\let\thedate\@date
\makeatother

\pagestyle{fancy}
\fancyhf{}
\rhead{\theauthor}
\lhead{\thetitle}
\cfoot{\thepage}
%%%%%%%%%%%%%%%%%%%%%%%%%%%%%%%%%%%%%%%%%%%%
\begin{document}

%%%%%%%%%%%%%%%%%%%%%%%%%%%%%%%%%%%%%%%%%%%%%%%%%%%%%%%%%%%%%%%%%%%%%%%%%%%%%%%%%%%%%%%%%

\begin{titlepage}
	\centering
    \vspace*{0.5 cm}
   % \includegraphics[scale = 0.075]{bsulogo.png}\\[1.0 cm]	% University Logo
\begin{center}    \textsc{\Large   ECE 351 - Section 52 \# }\\[2.0 cm]	\end{center}% University of Idaho
	\textsc{\Large  Final Project Report  }\\[0.5 cm]				% Course Code
	\rule{\linewidth}{0.2 mm} \\[0.4 cm]
	{ \huge \bfseries \thetitle}\\
	\rule{\linewidth}{0.2 mm} \\[1.5 cm]
	
	\begin{minipage}{0.4\textwidth}
		\begin{flushleft} \large
		%	\emph{Submitted To:}\\
		%	Name\\
          % Affiliation\\
           %contact info\\
			\end{flushleft}
			\end{minipage}~
			\begin{minipage}{0.4\textwidth}
            
			\begin{flushright} \large
			\emph{Submitted By :} \\
			Shujaea Aldousari  
		\end{flushright}
           
	\end{minipage}\\[2 cm]
	
%	\includegraphics[scale = 0.5]{PICMathLogo.png}
    
    
    
    
	
\end{titlepage}

%%%%%%%%%%%%%%%%%%%%%%%%%%%%%%%%%%%%%%%%%%%%%%%%%%%%%%%%%%%%%%%%%%%%%%%%%%%%%%%%%%%%%%%%%

\tableofcontents
\pagebreak

%%%%%%%%%%%%%%%%%%%%%%%%%%%%%%%%%%%%%%%%%%%%%%%%%%%%%%%%%%%%%%%%%%%%%%%%%%%%%%%%%%%%%%%%%
\renewcommand{\thesection}{\arabic{section}}
\section{Introduction}

In the final project, it is about working with an aircraft company. The signal contained with AC voltage wave form was given in range of 1.8kHz≤ f ≤ 2.0kHz. Also, the positioning system shares a ground connection with a switching amplifier, It is predicted that considerable noise at high frequency.  The input signal is affected by low frequency noise due to a small vibration that are generated from the building ventilation.






\section{equation}
We had to use the transfer function H(s)
$$H(s)=\frac{(R/l)*s}{s^2+{(R/L)s}+{(1/LC)}} $$

Then we choose the cut-off frequency $$Fc1=1.6kHz$$
and $$Fc2= 2.2kHz

Wc1= 2piFc1= 3.2x10^3 pi rad/s

$$Wc2= 2piFc1= 4.4x10 pi rad/s

$$B= Wc2-Wc1= 2piFc1= 1.2x10^3 pi rad/s $$

choose R= 3k
$$B= \frac{R}{L}$$

$$ L = \frac{R}{B} = \frac{3x10^3}{1.2x10^3}

$$= 0.84

$$Wo= \sqrt{Wc1*Wc2} = 3.75x10^3 pi Rad/s

$$Wo= \frac{1}{\sqrt{Lc}}$$

$$Wo^2= \frac{1}{LC}$$

$$ C = \frac{1}{LWo^2}= $$

$$ \frac{1}{(.8)(3.75x10^3 PI)^2} = 9 nF$$

so, we will have R = 3K ohm

 L= 0.8 H 

C= 9nF



$$


\section{Methodology}


To start off, we first list all the library needed for this lab which contains most of the things that we used during this course. Then we had to apply and use our user defined fft function that was on one of the previous labs then use it in this project. FFT function is used to transform the signal from time domain to frequncey domain. This will result us with frequency and magnitudes along with the signal which was the main idea of the first task is to read the Noisy Signal and use FFT function to do frequency analysis by plotting it with different ranges of frequencies. Secondly, In the next task we were asked to design analog filter and remove the low frequency noise and high frequency noise. After having the equations that listed above to find the components values (R,L,C). I can work in getting the plots for bode plots of the filter, then I changed the transfer function from S domain to Z domain to apply the design filter with the input sensor signal. In order to get the filterization, we have to use the sig.billinar which is the reason why we changed the function into Z-domain. Finally, We used FFT function to transform the filter signal to frequency domain and plot the signal with different range of frequency. 





\section{Result}

\includegraphics[scale=0.5]{1.png} 


\includegraphics[scale=0.5]{2.png} 


\includegraphics[scale=0.5]{3.png} 


\includegraphics[scale=0.5]{4.png} 

\includegraphics[scale=0.5]{5.png} 


\includegraphics[scale=0.5]{6.png} 


\includegraphics[scale=0.5]{7.png} 


\includegraphics[scale=0.5]{8.png} 

\includegraphics[scale=0.5]{9.png}


\includegraphics[scale=0.5]{10.png} 


\includegraphics[scale=0.5]{11.png} 

The values that we read is not zero, it is between 1800 Hz and 2000 Hz

\includegraphics[scale=0.5]{12.png} 

This is the low frequency signal that contains small frequency noise.

\includegraphics[scale=0.5]{13.png} 


\includegraphics[scale=0.5]{14.png} 














\section{Questions}

1.Earlier this semester, you were asked what you personally wanted to get out of taking this
course. Do you feel like that personal goal was met? Why or why not?

Yes, I feel like I have reached my goals but I still need to have improves on my skills. 

2. Please fill out the course feedback survey, I will read every word and very much appreciate
the feedback.

I will do that.

3. Good luck in the rest of your education and career!

Thank You Katie, and you too. Good luck for you.

\section{conclusion}

At the beginning, we deal with position measurement sensor as well as low and high frequency noise. We had to build our skills that we developed into this project. Through out the project, we read the measure mt position then we have analyzed it at frequency domain. Then we have plotted it both in time domain and frequency domain. Then, we have designed RLC filter to eliminate the low frequency noise and high frequency noise to get the desired signals. 


\newpage



\end{document}

%This template was created by Roza Aceska.