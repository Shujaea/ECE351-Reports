%%%%%%%%%%%%%%%%%%%%%%%%%%%%%%%%%%%%%%%%%%%
%%% DOCUMENT PREAMBLE %%%
\documentclass[12pt]{report}
\usepackage[english]{babel}
%\usepackage{natbib}
\usepackage{url}
\usepackage[utf8x]{inputenc}
\usepackage{amsmath}
\usepackage{graphicx}
\graphicspath{{images/}}
\usepackage{parskip}
\usepackage{fancyhdr}
\usepackage{vmargin}
\usepackage{listings}
\usepackage{hyperref}
\usepackage{xcolor}

\definecolor{codegreen}{rgb}{0,0.6,0}
\definecolor{codegray}{rgb}{0.5,0.5,0.5}
\definecolor{codeblue}{rgb}{0,0,0.95}
\definecolor{backcolour}{rgb}{0.95,0.95,0.92}

\lstdefinestyle{mystyle}{
    backgroundcolor=\color{backcolour},   
    commentstyle=\color{codegreen},
    keywordstyle=\color{codeblue},
    numberstyle=\tiny\color{codegray},
    stringstyle=\color{codegreen},
    basicstyle=\ttfamily\footnotesize,
    breakatwhitespace=false,         
    breaklines=true,                 
    captionpos=b,                    
    keepspaces=true,                 
    numbers=left,                    
    numbersep=5pt,                  
    showspaces=false,                
    showstringspaces=false,
    showtabs=false,                  
    tabsize=2
}
 
\lstset{style=mystyle}

\setmarginsrb{3 cm}{2.5 cm}{3 cm}{2.5 cm}{1 cm}{1.5 cm}{1 cm}{1.5 cm}

\title{Lab 9}								
% Title
\author{ Shujaea Aldousari}						
% Author
\date{Nov 3, 2021}
% Date

\makeatletter
\let\thetitle\@title
\let\theauthor\@author
\let\thedate\@date
\makeatother

\pagestyle{fancy}
\fancyhf{}
\rhead{\theauthor}
\lhead{\thetitle}
\cfoot{\thepage}
%%%%%%%%%%%%%%%%%%%%%%%%%%%%%%%%%%%%%%%%%%%%
\begin{document}

%%%%%%%%%%%%%%%%%%%%%%%%%%%%%%%%%%%%%%%%%%%%%%%%%%%%%%%%%%%%%%%%%%%%%%%%%%%%%%%%%%%%%%%%%

\begin{titlepage}
	\centering
    \vspace*{0.5 cm}
   % \includegraphics[scale = 0.075]{bsulogo.png}\\[1.0 cm]	% University Logo
\begin{center}    \textsc{\Large   ECE 351 - Section 52 \# }\\[2.0 cm]	\end{center}% University of Idaho
	\textsc{\Large Lab 9 Report  }\\[0.5 cm]				% Course Code
	\rule{\linewidth}{0.2 mm} \\[0.4 cm]
	{ \huge \bfseries \thetitle}\\
	\rule{\linewidth}{0.2 mm} \\[1.5 cm]
	
	\begin{minipage}{0.4\textwidth}
		\begin{flushleft} \large
		%	\emph{Submitted To:}\\
		%	Name\\
          % Affiliation\\
           %contact info\\
			\end{flushleft}
			\end{minipage}~
			\begin{minipage}{0.4\textwidth}
            
			\begin{flushright} \large
			\emph{Submitted By :} \\
			Shujaea Aldousari  
		\end{flushright}
           
	\end{minipage}\\[2 cm]
	
%	\includegraphics[scale = 0.5]{PICMathLogo.png}
    
    
    
    
	
\end{titlepage}

%%%%%%%%%%%%%%%%%%%%%%%%%%%%%%%%%%%%%%%%%%%%%%%%%%%%%%%%%%%%%%%%%%%%%%%%%%%%%%%%%%%%%%%%%

\tableofcontents
\pagebreak

%%%%%%%%%%%%%%%%%%%%%%%%%%%%%%%%%%%%%%%%%%%%%%%%%%%%%%%%%%%%%%%%%%%%%%%%%%%%%%%%%%%%%%%%%
\renewcommand{\thesection}{\arabic{section}}
\section{Introduction}

In this lab, we were working on fast Fourier transforms by using Python. After creating our user-defined function for fast Fourier transforms , we were able to analyze the plots for the signal.






\section{Part 1}

In the first part, We. were given the code which helped us during the lab, however, we had to fix it and add more details. We created our user defined function to apply fast Fourier transforms for the signal, then we had to work with the first signal until the last task.

    






\includegraphics[scale=0.5]{1.png} 
\includegraphics[scale=0.5]{2.png} 
\includegraphics[scale=0.5]{3.png}
\includegraphics[scale=0.5]{4.png}
\includegraphics[scale=0.5]{5.png}
\includegraphics[scale=0.5]{6.png}
\includegraphics[scale=0.5]{7.png} 
  caption{Caption}

  end{figure}







\section{Questions}
What happens if fs is lower? If it is higher? fs in your report must span a few orders of
magnitude.

if ffs lower, there will be interference's in the signal. However, if it is higher, it will be more accurate.

2. What difference does eliminating the small phase magnitudes make?

it will make it clearer, and we can specify the phasor

3. Verify your results from Tasks 1 and 2 using the Fourier transforms of cosine and sine. Explain why your results are correct. You will need the transforms in terms of Hz, not rad/s.
For example, the Fourier transform of cosine (in Hz) is:
F {cos (2πf0t)} =
1
2
[δ (f − f0) + δ (f + f0)]

Based on task 1 and 2, we got two inputs which they agree with the result to get two inputs in the frequency f0

4. Leave any feedback on the clarity of lab tasks, expectations, and deliverables






\section{conclusion}
In this lab, we learned about the fast Fourier Series transform . At the very begging, we were asked to  create our user defined function to apply fast Fourier transforms for the signal. For the first signal, it was cos(2*pi) which we plotted the signal and the magnitude with out limit and with limit frequency. Also we plotted the phase for the same signal with limit frequency and with out limit frequency. Repeatedly, we did the same thing for the 5sin(2πt) as well as 2cos((2π · 2t) − 2) + sin2
((2π · 6t) + 3). However, in task 4, it's found that the phase plot was not clear so, we editted the fft to make the phase plot be clear. The way it has been edited is by making the phase equal to zero when the magnitude is smaller than 1e-10. After that we ran the code for the three function.
For task 5, we did fast Fourier transforms to the square wave that we did in the previous lab for N = 8 and T = 8. At the end, this lab was a great lab for dealing with fast Fourier transforms with different signals.


\newpage



\end{document}

%This template was created by Roza Aceska.