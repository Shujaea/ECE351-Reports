%%%%%%%%%%%%%%%%%%%%%%%%%%%%%%%%%%%%%%%%%%%
%%% DOCUMENT PREAMBLE %%%
\documentclass[12pt]{report}
\usepackage[english]{babel}
%\usepackage{natbib}
\usepackage{url}
\usepackage[utf8x]{inputenc}
\usepackage{amsmath}
\usepackage{graphicx}
\graphicspath{{images/}}
\usepackage{parskip}
\usepackage{fancyhdr}
\usepackage{vmargin}
\usepackage{listings}
\usepackage{hyperref}
\usepackage{xcolor}

\definecolor{codegreen}{rgb}{0,0.6,0}
\definecolor{codegray}{rgb}{0.5,0.5,0.5}
\definecolor{codeblue}{rgb}{0,0,0.95}
\definecolor{backcolour}{rgb}{0.95,0.95,0.92}

\lstdefinestyle{mystyle}{
    backgroundcolor=\color{backcolour},   
    commentstyle=\color{codegreen},
    keywordstyle=\color{codeblue},
    numberstyle=\tiny\color{codegray},
    stringstyle=\color{codegreen},
    basicstyle=\ttfamily\footnotesize,
    breakatwhitespace=false,         
    breaklines=true,                 
    captionpos=b,                    
    keepspaces=true,                 
    numbers=left,                    
    numbersep=5pt,                  
    showspaces=false,                
    showstringspaces=false,
    showtabs=false,                  
    tabsize=2
}
 
\lstset{style=mystyle}

\setmarginsrb{3 cm}{2.5 cm}{3 cm}{2.5 cm}{1 cm}{1.5 cm}{1 cm}{1.5 cm}

\title{Lab6}								
% Title
\author{ Shujaea Aldousari}						
% Author
\date{Oct 13, 2021}
% Date

\makeatletter
\let\thetitle\@title
\let\theauthor\@author
\let\thedate\@date
\makeatother

\pagestyle{fancy}
\fancyhf{}
\rhead{\theauthor}
\lhead{\thetitle}
\cfoot{\thepage}
%%%%%%%%%%%%%%%%%%%%%%%%%%%%%%%%%%%%%%%%%%%%
\begin{document}

%%%%%%%%%%%%%%%%%%%%%%%%%%%%%%%%%%%%%%%%%%%%%%%%%%%%%%%%%%%%%%%%%%%%%%%%%%%%%%%%%%%%%%%%%

\begin{titlepage}
	\centering
    \vspace*{0.5 cm}
   % \includegraphics[scale = 0.075]{bsulogo.png}\\[1.0 cm]	% University Logo
\begin{center}    \textsc{\Large   ECE 351 - Section 52 \# }\\[2.0 cm]	\end{center}% University of Idaho
	\textsc{\Large Lab 6 Report  }\\[0.5 cm]				% Course Code
	\rule{\linewidth}{0.2 mm} \\[0.4 cm]
	{ \huge \bfseries \thetitle}\\
	\rule{\linewidth}{0.2 mm} \\[1.5 cm]
	
	\begin{minipage}{0.4\textwidth}
		\begin{flushleft} \large
		%	\emph{Submitted To:}\\
		%	Name\\
          % Affiliation\\
           %contact info\\
			\end{flushleft}
			\end{minipage}~
			\begin{minipage}{0.4\textwidth}
            
			\begin{flushright} \large
			\emph{Submitted By :} \\
			Shujaea Aldousari  
		\end{flushright}
           
	\end{minipage}\\[2 cm]
	
%	\includegraphics[scale = 0.5]{PICMathLogo.png}
    
    
    
    
	
\end{titlepage}

%%%%%%%%%%%%%%%%%%%%%%%%%%%%%%%%%%%%%%%%%%%%%%%%%%%%%%%%%%%%%%%%%%%%%%%%%%%%%%%%%%%%%%%%%

\tableofcontents
\pagebreak

%%%%%%%%%%%%%%%%%%%%%%%%%%%%%%%%%%%%%%%%%%%%%%%%%%%%%%%%%%%%%%%%%%%%%%%%%%%%%%%%%%%%%%%%%
\renewcommand{\thesection}{\arabic{section}}
\section{Introduction}
 

In this lab, we learned about how to use scipy.signal.residue() functions which it helped us to perform the partial fraction expansion, Also it was the main purpose of this lab. Before the lab, we were asked to work on the pre-lab in order to find the transfer function. Next, we had to find the output 




\section{Part 1}

We were asked to find/plot our user defined function based on the lab Manuel as well as with a help of the function we found in the pre-lab. For the next task, we were asked to get the plot from the step response function we found in the pre-lab by using the scipy.signal.residue(). We compared the hand calculated with the step response and as shown in the plots. Finally, in the last task we were asked to get the residue function to get partial fraction expanstion for the Y(s) function that we did in the pre-lab. The results of R,P,K as attached below 





\begin{lstlisting}[language=Python]



import numpy as np
import matplotlib . pyplot as plt
import scipy . signal as sig

steps = 1e-2 # Define step size
t = np. arange (0, 2 + steps , steps ) 

# user defined function
def step (t): 
    y = np. zeros (t. shape ) 
    for i in range ( len (t)): 
        if t[i] >= 0:  
            y[i] = 1
        else:
            y[i] = 0
    return y 
#Part 1
#Task 1
def yt(t):
    y = (0.5 - 0.5 * np.exp(-4*t) + np.exp(-6*t)) * step (t) 
    return y
#Task 2
numH = [1, 6 , 12]
denH = [1, 10 , 24]
tout, yout = sig. step (( numH , denH), T = t)

plt. figure ( figsize = (10 , 7))
plt. subplot (2, 1, 1)
plt. plot (t, yt(t))
plt. grid ()
plt. ylabel ('Hand Calculated')
plt. title (' Step Responce y(t)')
plt. subplot (2, 1, 2)
plt. plot (tout, yout)
plt. grid ()
plt. ylabel ('Built-in')
plt. xlabel ('t')

#Task 3
numY = [1, 6 , 12]
denY = [1, 10 , 24, 0]
R, P, K = sig.residue(numY , denY)
print("R= ", R)
print("P= ", P)
print("K= ", K)








\end{lstlisting}

\includegraphics{1.png}
\includegraphics{output part1.png}
\section{Part 2 }

In part 2, we were still working on the residue function to perform partial fraction expansion.  

\begin{lstlisting}[language=Python]

#Part 2
#Task 1
numY2 = 25250
denY2 = [1, 18, 218, 2036, 9085, 25250, 0]
R2, P2, K2 = sig.residue(numY2 , denY2)
print("R2= ", R2)
print("P2= ", P2)
print("K2= ", K2)

#Task 2
t = np. arange (0, 4.5 + steps , steps ) 
L = len(R2) 
y2 = np. zeros(t. shape)
for i in range (L):
    MAG_R = np.abs(R2[i])
    Ang_R = np.angle(R2[i])
    Real_P = P2[i].real
    Img_P = P2[i].imag
    y2 += MAG_R * np.exp(Real_P*t) * np.cos(Img_P*t + Ang_R) *step(t)

#Task 3
numH2 = 25250
denH2 = [1, 18, 218, 2036, 9085, 25250]
tout2, yout2 = sig. step (( numH2 , denH2), T = t)

plt. figure ( figsize = (10 , 7))
plt. subplot (2, 1, 1)
plt. plot (t, y2)
plt. grid ()
plt. ylabel ('Hand Calculated')
plt. title (' Step Responce y(t)')
plt. subplot (2, 1, 2)
plt. plot (tout2, yout2)
plt. grid ()
plt. ylabel ('Built-in')
plt. xlabel ('t')

plt. show ()


\end{lstlisting}

\includegraphics{2.png}
\includegraphics{output part2.png}




\section{Questions}
1. For a non-complex pole-residue term, you can still use the cosine method, explain why this
works.

2. Leave any feedback on the clarity of the expectations, instructions, and deliverables. 

Q1. for non-complex residue, we will get the magnitude and the angle. The magnitude will be positive and for the angle we have two options. The angle 0 for positive number, and angle 180 for negative number. However for cos method, cos zero will be positive and cos 180 will be negative.
for non-complex pole, we will get the real and imaginary parts. The real part will be the non-complex pole but the imaginary part will be zero and cos zero equal 1. Therefore, non-complex pole residue can be obtain by using cos method.

overall, the lab was clear enough and easy to work with. 




\section{conclusion}


In this lab, we learned how to use the residue function built in python. At the beginning of this lab, we plotted step response for the pre-lab system then we used step built in function for the same system as the pre-lab, also we make sure the plot we had in task 1 is same as task 2. After that, we used the residue function to find the partial fraction expansion and we check that the values we obtain in pre-lab are the same as the values we obtain by the residue function. Next, we used the residue function to get the partial fraction expansion for a complex system. Then we used these values to plot the step response for the complex system by using cos method. Finally, we used the step response to check the plot of the step response for task 2, and we ensure that the two plots are the same.

\newpage



\end{document}

%This template was created by Roza Aceska.