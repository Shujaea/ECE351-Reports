%%%%%%%%%%%%%%%%%%%%%%%%%%%%%%%%%%%%%%%%%%%
%%% DOCUMENT PREAMBLE %%%
\documentclass[12pt]{report}
\usepackage[english]{babel}
%\usepackage{natbib}
\usepackage{url}
\usepackage[utf8x]{inputenc}
\usepackage{amsmath}
\usepackage{graphicx}
\graphicspath{{images/}}
\usepackage{parskip}
\usepackage{fancyhdr}
\usepackage{vmargin}
\usepackage{listings}
\usepackage{hyperref}
\usepackage{xcolor}

\definecolor{codegreen}{rgb}{0,0.6,0}
\definecolor{codegray}{rgb}{0.5,0.5,0.5}
\definecolor{codeblue}{rgb}{0,0,0.95}
\definecolor{backcolour}{rgb}{0.95,0.95,0.92}

\lstdefinestyle{mystyle}{
    backgroundcolor=\color{backcolour},   
    commentstyle=\color{codegreen},
    keywordstyle=\color{codeblue},
    numberstyle=\tiny\color{codegray},
    stringstyle=\color{codegreen},
    basicstyle=\ttfamily\footnotesize,
    breakatwhitespace=false,         
    breaklines=true,                 
    captionpos=b,                    
    keepspaces=true,                 
    numbers=left,                    
    numbersep=5pt,                  
    showspaces=false,                
    showstringspaces=false,
    showtabs=false,                  
    tabsize=2
}
 
\lstset{style=mystyle}

\setmarginsrb{3 cm}{2.5 cm}{3 cm}{2.5 cm}{1 cm}{1.5 cm}{1 cm}{1.5 cm}

\title{Lab 11}								
% Title
\author{ Shujaea Aldousari}						
% Author
\date{Nov 17, 2021}
% Date

\makeatletter
\let\thetitle\@title
\let\theauthor\@author
\let\thedate\@date
\makeatother

\pagestyle{fancy}
\fancyhf{}
\rhead{\theauthor}
\lhead{\thetitle}
\cfoot{\thepage}
%%%%%%%%%%%%%%%%%%%%%%%%%%%%%%%%%%%%%%%%%%%%
\begin{document}

%%%%%%%%%%%%%%%%%%%%%%%%%%%%%%%%%%%%%%%%%%%%%%%%%%%%%%%%%%%%%%%%%%%%%%%%%%%%%%%%%%%%%%%%%

\begin{titlepage}
	\centering
    \vspace*{0.5 cm}
   % \includegraphics[scale = 0.075]{bsulogo.png}\\[1.0 cm]	% University Logo
\begin{center}    \textsc{\Large   ECE 351 - Section 52 \# }\\[2.0 cm]	\end{center}% University of Idaho
	\textsc{\Large Lab 11 Report  }\\[0.5 cm]				% Course Code
	\rule{\linewidth}{0.2 mm} \\[0.4 cm]
	{ \huge \bfseries \thetitle}\\
	\rule{\linewidth}{0.2 mm} \\[1.5 cm]
	
	\begin{minipage}{0.4\textwidth}
		\begin{flushleft} \large
		%	\emph{Submitted To:}\\
		%	Name\\
          % Affiliation\\
           %contact info\\
			\end{flushleft}
			\end{minipage}~
			\begin{minipage}{0.4\textwidth}
            
			\begin{flushright} \large
			\emph{Submitted By :} \\
			Shujaea Aldousari  
		\end{flushright}
           
	\end{minipage}\\[2 cm]
	
%	\includegraphics[scale = 0.5]{PICMathLogo.png}
    
    
    
    
	
\end{titlepage}

%%%%%%%%%%%%%%%%%%%%%%%%%%%%%%%%%%%%%%%%%%%%%%%%%%%%%%%%%%%%%%%%%%%%%%%%%%%%%%%%%%%%%%%%%

\tableofcontents
\pagebreak

%%%%%%%%%%%%%%%%%%%%%%%%%%%%%%%%%%%%%%%%%%%%%%%%%%%%%%%%%%%%%%%%%%%%%%%%%%%%%%%%%%%%%%%%%
\renewcommand{\thesection}{\arabic{section}}
\section{Introduction}

In this lab, we were working on Analyzing a discrete system using Python’s built-in functions and a function developed by Christopher Felton. Below are equations that we applied and solved in order to work to keep working on the lab.


Task 1
$$y[K]-10yy[k-1]+16y[k-2]=2x[k]-40x[k-1]$$

Apply Z- transform
$$Y(z)-10^-1Y(z)+16z^-2Y(z)=2x(z)-40z^-1x(z)$$
$$Y(z)[1-10z^-1+16z^-2]=x(z)[2-40z^-1]$$
$$H(z)=\frac{Y(z)}{X(z)}=\frac{(2-40z^-1)}{1-10z^-1+16z^-2}$$
$$H(z)=\frac{Y(z)}{X(z)}=\frac{2z(z-20)}{z^2-10z+16}$$
$$=\frac{(2z^2-40z)}{z^2-10z+16}$$

Task 2 
$$H(z) =\frac{2z(z-20)}{z^2-10z+16}$$
$$\frac{(H(z)}{z}= \frac{2(z-20)}{z^2-10z+16}= \frac{2(z-20)}{(z-2)(z-8)}= \frac{(A)}{z-2}= \frac{(B)}{z-8}$$


$$A= 6 $$
$$B= -4 $$
$$\frac{(H(z)}{z}= \frac{(6)}{z-2} + \frac{(-4)}{(z-8)}$$
$$\frac{(H(z)}{z}= \frac{(6z)}{z-2} + \frac{(-4z)}{(z-8)}$$
Apply Inversve z-transform
$$h(k)=[6*2^k-4*8^k]u[k]$$


Task 3-5:
We were asked to use Use scipy.signal.residuez() into python. So, we had to check for the response function, Also to get the poles and the residues for the partial fraction expansion of H(z). In the next task, we were given an example for the zplane() function that been used so we can get the poles and zeros plot for H(z). Finaly, we were asked in the last task to plot the magnatitue and the phase for our H(z) by using scipy.signal.freqz().



\section{Result}
\begin{lstlisting}[language=Python]

import numpy as np
import scipy.signal as sig
import matplotlib.pyplot as plt


def zplane(b, a, filename = None):
    """ Plot the complex z- plane given a transfer function """
    
    from matplotlib import patches
    
    # get a figure / plot
    ax = plt.subplot(1, 1, 1)
    plt.title('Figure 1: Pole-Zero Plot for H(z)')
    
    # create the unit circle
    uc = patches.Circle((0, 0), radius = 1, fill = False, color = 'black', 
                        ls = 'dashed')
    ax.add_patch(uc)
    
    # the coefficients are less than 1 , normalize the coefficients
    if np.max(b) > 1:
        kn = np.max(b)
        b = np.array(b)/float(kn)
    else:
        kn = 1
    
    if np.max(a) > 1:
        kd = np.max(a)
        a = np.array(a)/float(kd)
    else:
        kd = 1
    
    # get the poles and zeros
    p = np.roots(a)
    z = np.roots(b)
    k = kn/float(kd)
    
    # plot the zeros and set marker properties
    t1 = plt.plot(z.real, z.imag, 'o', ms = 10, label = 'Zeros')
    plt.setp(t1, markersize = 10.0, markeredgewidth = 1.0)
    
    # plot the poles and set marker properties
    t2 = plt.plot(p.real, p.imag, 'x', ms = 10, label = 'Poles')
    plt.setp(t2, markersize = 12.0, markeredgewidth = 3.0)
    
    ax.spines['left'].set_position('center')
    ax.spines['bottom'].set_position('center')
    ax.spines['right'].set_visible(False)
    ax.spines['top'].set_visible(False)
    
    plt.legend()
    
    # set the ticks
    
    # r = 1.5; plt. axis ( ’ scaled ’); plt. axis ([ -r, r, -r, r])
    # ticks = [ -1 , -.5 , .5 , 1]; plt. xticks ( ticks ); plt. yticks ( ticks )
    
    if filename is None:
        plt.show()
    else:
        plt.savefig(filename)
    
    return z, p, k

#Task 3 H(Z)/Z
NumH = [2, -40]
DenH = [1, -10, 16]

R, P, K = sig.residuez(NumH, DenH)

print("H(Z)/Z: ")
print("Residue= ", R)
print("Poles= ", P)
print("K= ", K)

#Task 4 H(Z)
NumHZ = [2, -40, 0]
DenHZ = [1, -10, 16]
Z, P, K = zplane(NumHZ, DenHZ)
print("H(Z) ")
print("Zeros= ", Z)
print("Poles= ", P)
print("K= ", K)

#Task 5
w, H = sig.freqz(NumHZ, DenHZ, whole = True)
f = w/(2*np.pi)
Hmag = np.absolute(H)
HmagDB = 20*np.log10(Hmag)
Hphase = np.angle(H)*180/np.pi

plt.figure(figsize = (7, 5))
plt.subplot(2, 1, 1)
plt.title('Frequency Responce')
plt.plot(f, HmagDB)
plt.grid()
plt.ylabel('Mag (dB)')

plt.subplot(2, 1, 2)
plt.plot(f, Hphase)
plt.grid()
plt.ylabel('Phase (deg)')
plt.xlabel('f (Hz)')

plt.show()

\end{lstlisting}

\includegraphics[scale=0.5]{1.png} 
\includegraphics[scale=0.5]{2.png} 
\includegraphics[scale=0.5]{3.png} 
\includegraphics[scale=0.5]{4.png} 









\section{Questions}

Looking at the plot generated in Task 4, is H(z) stable? Explain why or why not.

2. Leave any feedback on the clarity of lab tasks, expectations, and deliverables.


The poles are located at z = 2 and z = 8, and the region conversion does not contain the unit circle so the system is unstable


Overall, the lab was clear.



\section{conclusion}

In this lab, We deal with Z-transform. At the beginning, we found the frequency response H(z) from the differential equation. In the second task, we used partial fraction expansion to find h(k). In the third task, we used function residuez built in by python to check and verify the result that we found. After that, we used Z-plane function to find zeros and plot them on z-plane. Finally, we used freqz function to plot the magnitude and the phase with respect to frequency. 




\newpage



\end{document}

%This template was created by Roza Aceska.