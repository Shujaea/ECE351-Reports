%%%%%%%%%%%%%%%%%%%%%%%%%%%%%%%%%%%%%%%%%%%
%%% DOCUMENT PREAMBLE %%%
\documentclass[12pt]{report}
\usepackage[english]{babel}
%\usepackage{natbib}
\usepackage{url}
\usepackage[utf8x]{inputenc}
\usepackage{amsmath}
\usepackage{graphicx}
\graphicspath{{images/}}
\usepackage{parskip}
\usepackage{fancyhdr}
\usepackage{vmargin}
\usepackage{listings}
\usepackage{hyperref}
\usepackage{xcolor}

\definecolor{codegreen}{rgb}{0,0.6,0}
\definecolor{codegray}{rgb}{0.5,0.5,0.5}
\definecolor{codeblue}{rgb}{0,0,0.95}
\definecolor{backcolour}{rgb}{0.95,0.95,0.92}

\lstdefinestyle{mystyle}{
    backgroundcolor=\color{backcolour},   
    commentstyle=\color{codegreen},
    keywordstyle=\color{codeblue},
    numberstyle=\tiny\color{codegray},
    stringstyle=\color{codegreen},
    basicstyle=\ttfamily\footnotesize,
    breakatwhitespace=false,         
    breaklines=true,                 
    captionpos=b,                    
    keepspaces=true,                 
    numbers=left,                    
    numbersep=5pt,                  
    showspaces=false,                
    showstringspaces=false,
    showtabs=false,                  
    tabsize=2
}
 
\lstset{style=mystyle}

\setmarginsrb{3 cm}{2.5 cm}{3 cm}{2.5 cm}{1 cm}{1.5 cm}{1 cm}{1.5 cm}

\title{Lab4}								
% Title
\author{ Shujaea Aldousari}						
% Author
\date{Sep 29, 2021}
% Date

\makeatletter
\let\thetitle\@title
\let\theauthor\@author
\let\thedate\@date
\makeatother

\pagestyle{fancy}
\fancyhf{}
\rhead{\theauthor}
\lhead{\thetitle}
\cfoot{\thepage}
%%%%%%%%%%%%%%%%%%%%%%%%%%%%%%%%%%%%%%%%%%%%
\begin{document}

%%%%%%%%%%%%%%%%%%%%%%%%%%%%%%%%%%%%%%%%%%%%%%%%%%%%%%%%%%%%%%%%%%%%%%%%%%%%%%%%%%%%%%%%%

\begin{titlepage}
	\centering
    \vspace*{0.5 cm}
   % \includegraphics[scale = 0.075]{bsulogo.png}\\[1.0 cm]	% University Logo
\begin{center}    \textsc{\Large   ECE 351 - Section 52 \# }\\[2.0 cm]	\end{center}% University of Idaho
	\textsc{\Large Lab 4 Report  }\\[0.5 cm]				% Course Code
	\rule{\linewidth}{0.2 mm} \\[0.4 cm]
	{ \huge \bfseries \thetitle}\\
	\rule{\linewidth}{0.2 mm} \\[1.5 cm]
	
	\begin{minipage}{0.4\textwidth}
		\begin{flushleft} \large
		%	\emph{Submitted To:}\\
		%	Name\\
          % Affiliation\\
           %contact info\\
			\end{flushleft}
			\end{minipage}~
			\begin{minipage}{0.4\textwidth}
            
			\begin{flushright} \large
			\emph{Submitted By :} \\
			Shujaea Aldousari  
		\end{flushright}
           
	\end{minipage}\\[2 cm]
	
%	\includegraphics[scale = 0.5]{PICMathLogo.png}
    
    
    
    
	
\end{titlepage}

%%%%%%%%%%%%%%%%%%%%%%%%%%%%%%%%%%%%%%%%%%%%%%%%%%%%%%%%%%%%%%%%%%%%%%%%%%%%%%%%%%%%%%%%%

\tableofcontents
\pagebreak

%%%%%%%%%%%%%%%%%%%%%%%%%%%%%%%%%%%%%%%%%%%%%%%%%%%%%%%%%%%%%%%%%%%%%%%%%%%%%%%%%%%%%%%%%
\renewcommand{\thesection}{\arabic{section}}
\section{Introduction}
 

In this lab, we were introduced with system step response as well as keep continue working on convolution.




\section{Part 1}

We were asked to create a signals by using the user defined function. In this lab, we had 3 signals which they are
h1(t) = e −2t [u(t) − u(t − 3)] 
h2(t) = u(t − 2) − u(t − 6) 
h3(t) = cos(ω0t)u(t)
We created the codes and plotted the 3 functions in a single figure, with step of −10 ≤ t ≤ 10, after running the code. Will have a three functions that contains the following a graph of h1,h2 and h3.





\begin{lstlisting}[language=Python]

import numpy as np
import matplotlib . pyplot as plt
import math


steps = 1e-2 # Define step size
t = np. arange (-10, 10 + steps , steps ) 

# user defined function
def step (t): 
    y = np. zeros (t. shape ) 
    for i in range ( len (t)): 
        if t[i] >= 0:  
            y[i] = 1
        else:
            y[i] = 0
    return y 

def ramp (t): 
    y = np. zeros (t. shape ) 
    for i in range ( len (t)): 
        if t[i] >= 0:  
            y[i] = t[i]
        else:
            y[i] = 0
    return y 

def h1(t):
    y = np.exp(-2*t) *(step (t) - step (t-3))
    return y

def h2(t):
    y = step (t-2) - step (t-6)
    return y

def h3(t):
    y =  np.cos(2 * math.pi *0.25 *t) * step (t)
    return y

plt. figure ( figsize = (10 , 7))
plt. subplot (3, 1, 1)
plt. plot (t, h1(t))
plt. grid ()
plt. ylabel ('h1(t)')
plt. title (' Task 1')
plt. subplot (3, 1, 2)
plt. plot (t, h2(t))
plt. grid ()
plt. ylabel ('h2(t)')
plt. subplot (3, 1, 3)
plt. plot (t, h3(t))
plt. grid ()
plt. ylabel ('h3(t)')
plt. xlabel ('t')

plt. show ()




\end{lstlisting}

\includegraphics{1.png}

\section{Part 2 }

We were asked to plot step responses of the 3 functions that we used in the previous part. The code for the convolution part has been edited. After creating the codes, we had 6 graphs as shown in the results. We also used the built-in convolution to check the results that we made.

\begin{lstlisting}[language=Python]
import numpy as np
import matplotlib . pyplot as plt
import math
import scipy . signal as sig


steps = 1e-2 # Define step size
t = np. arange (-10, 10 + steps , steps ) 
w0 = 2 * math.pi *0.25 

# user defined function
def step (t): 
    y = np. zeros (t. shape ) 
    for i in range ( len (t)): 
        if t[i] >= 0:  
            y[i] = 1
        else:
            y[i] = 0
    return y 

def ramp (t): 
    y = np. zeros (t. shape ) 
    for i in range ( len (t)): 
        if t[i] >= 0:  
            y[i] = t[i]
        else:
            y[i] = 0
    return y 

def h1(t):
    y = np.exp(-2*t) *(step (t) - step (t-3))
    return y

def h2(t):
    y = step (t-2) - step (t-6)
    return y

def h3(t):
    y =  np.cos(w0 *t) * step (t)
    return y

def conv(fu1, fu2):
    conf1 = len (fu1 )
    conf2 = len (fu2 )
    f1E = np. append (fu1 , np. zeros ((1 , conf2 -1) ))
    f2E = np. append (fu2 , np. zeros ((1 , conf1 -1) ))
    Value = np. zeros (f1E . shape )
    for i in range ( conf2 + conf1 - 2):
        Value [i] = 0
        for j in range ( conf1 ):
            if(i - j + 1 > 0):
                Value [i] += f1E [j]* f2E [i - j + 1]
    return Value

y1 = conv(h1(t),step (t)) #user-created convolution
Y1 = sig.convolve(h1(t),step (t)) #Built-in convolution

y2 = conv(h2(t),step (t)) #user-created convolution
Y2 = sig.convolve(h2(t),step (t)) #Built-in convolution

y3 = conv(h3(t),step (t)) #user-created convolution
Y3 = sig.convolve(h3(t),step (t)) #Built-in convolution

t = np.linspace(-10, 10, len(y1))


plt. figure ( figsize = (10 , 7))
plt. subplot (2, 1, 1)
plt. plot (t, y1)
plt. grid ()
plt. ylabel ('y1(t)')
plt. title (' First responce')
plt. subplot (2, 1, 2)
plt. plot (t, Y1)
plt. grid ()
plt. ylabel ('Y1(t)')
plt. xlabel ('t')

plt. figure ( figsize = (10 , 7))
plt. subplot (2, 1, 1)
plt. plot (t, y2)
plt. grid ()
plt. ylabel ('y2(t)')
plt. title (' Second responce')
plt. subplot (2, 1, 2)
plt. plot (t, Y2)
plt. grid ()
plt. ylabel ('Y2(t)')
plt. xlabel ('t')

plt. figure ( figsize = (10 , 7))
plt. subplot (2, 1, 1)
plt. plot (t, y3)
plt. grid ()
plt. ylabel ('y3(t)')
plt. title (' Third responce')
plt. subplot (2, 1, 2)
plt. plot (t, Y3)
plt. grid ()
plt. ylabel ('Y3(t)')
plt. xlabel ('t')

#hand Calculated step responce
y1c = 0.5 * (1 - np.exp(-2*t)) * (step (t) - step (t-3)) + 0.5 * step (t-3)
y2c = (t-2) * (step (t-2) - step (t-6)) + 4 * step (t-6)
y3c = np.cos(w0 *t) * step (t) / w0

plt. figure ( figsize = (10 , 7))
plt. subplot (3, 1, 1)
plt. plot (t, y1c)
plt. grid ()
plt. ylabel ('y1c(t)')
plt. title (' Step Respnoce (Hand Calculated)')
plt. subplot (3, 1, 2)
plt. plot (t, y2c)
plt. grid ()
plt. ylabel ('y2c(t)')
plt. subplot (3, 1, 3)
plt. plot (t, y3c)
plt. grid ()
plt. ylabel ('y3c(t)')
plt. xlabel ('t')

plt. show ()

\end{lstlisting}

\includegraphics{2.png}
\includegraphics{3.png}
\includegraphics{4.png}


\section{Questions}

Leave any feedback on the clarity of lab tasks, expectations, and deliverable.
This lab gives a solid idea about the transfer functions and step response as well as how we compute it in the step responses.



\section{conclusion}


Overall, the main idea of this lab to introduce the transfer functions as well as to be more familiar with convolution.

\newpage



\end{document}

%This template was created by Roza Aceska.