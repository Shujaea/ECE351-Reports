%%%%%%%%%%%%%%%%%%%%%%%%%%%%%%%%%%%%%%%%%%%
%%% DOCUMENT PREAMBLE %%%
\documentclass[12pt]{report}
\usepackage[english]{babel}
%\usepackage{natbib}
\usepackage{url}
\usepackage[utf8x]{inputenc}
\usepackage{amsmath}
\usepackage{graphicx}
\graphicspath{{images/}}
\usepackage{parskip}
\usepackage{fancyhdr}
\usepackage{vmargin}
\usepackage{listings}
\usepackage{hyperref}
\usepackage{xcolor}

\definecolor{codegreen}{rgb}{0,0.6,0}
\definecolor{codegray}{rgb}{0.5,0.5,0.5}
\definecolor{codeblue}{rgb}{0,0,0.95}
\definecolor{backcolour}{rgb}{0.95,0.95,0.92}

\lstdefinestyle{mystyle}{
    backgroundcolor=\color{backcolour},   
    commentstyle=\color{codegreen},
    keywordstyle=\color{codeblue},
    numberstyle=\tiny\color{codegray},
    stringstyle=\color{codegreen},
    basicstyle=\ttfamily\footnotesize,
    breakatwhitespace=false,         
    breaklines=true,                 
    captionpos=b,                    
    keepspaces=true,                 
    numbers=left,                    
    numbersep=5pt,                  
    showspaces=false,                
    showstringspaces=false,
    showtabs=false,                  
    tabsize=2
}
 
\lstset{style=mystyle}

\setmarginsrb{3 cm}{2.5 cm}{3 cm}{2.5 cm}{1 cm}{1.5 cm}{1 cm}{1.5 cm}

\title{Lab 8}								
% Title
\author{ Shujaea Aldousari}						
% Author
\date{Oct 26, 2021}
% Date

\makeatletter
\let\thetitle\@title
\let\theauthor\@author
\let\thedate\@date
\makeatother

\pagestyle{fancy}
\fancyhf{}
\rhead{\theauthor}
\lhead{\thetitle}
\cfoot{\thepage}
%%%%%%%%%%%%%%%%%%%%%%%%%%%%%%%%%%%%%%%%%%%%
\begin{document}

%%%%%%%%%%%%%%%%%%%%%%%%%%%%%%%%%%%%%%%%%%%%%%%%%%%%%%%%%%%%%%%%%%%%%%%%%%%%%%%%%%%%%%%%%

\begin{titlepage}
	\centering
    \vspace*{0.5 cm}
   % \includegraphics[scale = 0.075]{bsulogo.png}\\[1.0 cm]	% University Logo
\begin{center}    \textsc{\Large   ECE 351 - Section 52 \# }\\[2.0 cm]	\end{center}% University of Idaho
	\textsc{\Large Lab 8 Report  }\\[0.5 cm]				% Course Code
	\rule{\linewidth}{0.2 mm} \\[0.4 cm]
	{ \huge \bfseries \thetitle}\\
	\rule{\linewidth}{0.2 mm} \\[1.5 cm]
	
	\begin{minipage}{0.4\textwidth}
		\begin{flushleft} \large
		%	\emph{Submitted To:}\\
		%	Name\\
          % Affiliation\\
           %contact info\\
			\end{flushleft}
			\end{minipage}~
			\begin{minipage}{0.4\textwidth}
            
			\begin{flushright} \large
			\emph{Submitted By :} \\
			Shujaea Aldousari  
		\end{flushright}
           
	\end{minipage}\\[2 cm]
	
%	\includegraphics[scale = 0.5]{PICMathLogo.png}
    
    
    
    
	
\end{titlepage}

%%%%%%%%%%%%%%%%%%%%%%%%%%%%%%%%%%%%%%%%%%%%%%%%%%%%%%%%%%%%%%%%%%%%%%%%%%%%%%%%%%%%%%%%%

\tableofcontents
\pagebreak

%%%%%%%%%%%%%%%%%%%%%%%%%%%%%%%%%%%%%%%%%%%%%%%%%%%%%%%%%%%%%%%%%%%%%%%%%%%%%%%%%%%%%%%%%
\renewcommand{\thesection}{\arabic{section}}
\section{Introduction}

In this lab, we learned how to Fourier Series to approximate periodic time-domain signals.
 






\section{Part 1}

In the first part, we were asked to Input the expressions for ak and bk into Spyder. With The help of the pre-lab, we were able to get the numbers in Python. After that, we plotted the signal for the square wave When the Numbers of N changes. As long as N increases, we will have more accuracy.

    



\begin{lstlisting}[language=Python]

import numpy as np
import matplotlib.pyplot as plt

steps = 1e-4 # Define step size
t = np. arange (0, 20 + steps , steps ) 

#Task 1
a = np.zeros((4, 1))
b = np.zeros((4, 1))

for k in np.arange(1, 4):
    b[k] = 2* (1-np.cos(k*np.pi))/(k*np.pi)
    
print('a0 = ', a[0])
print('a1 = ', a[1])
print('b1 = ', b[1])
print('b2 = ', b[2])
print('b3 = ', b[3])

#Task2
# user defined function
def FS(N, T, t): 
    y = np. zeros (t. shape ) 
    for k in np.arange(1, N+1):
        b =  2*(1-np.cos(k*np.pi))/(k*np.pi)
        S = b*np.sin(k*2*np.pi*t/T)
        y += S
    return y 
T = 8

X1=FS(1, T, t)

X3=FS(3, T, t)

X15=FS(15, T, t)

X50=FS(50, T, t)

X150=FS(150, T, t)

X1500= FS(1500, T, t)

plt. figure ( figsize = (10 , 7))
plt. subplot (3, 1, 1)
plt. plot (t, X1)
plt. grid ()
plt. ylabel ('x1(t)')
plt. title (' Fourier Series Approximation of a Square Wave')
plt. subplot (3, 1, 2)
plt. plot (t, X3)
plt. grid ()
plt. ylabel ('X3(t)')
plt. subplot (3, 1, 3)
plt. plot (t, X15)
plt. grid ()
plt. ylabel ('X15(t)')
plt. xlabel ('t')

plt. figure ( figsize = (10 , 7))
plt. subplot (3, 1, 1)
plt. plot (t, X50)
plt. grid ()
plt. ylabel ('x50(t)')
plt. title (' Fourier Series Approximation of a Square Wave')
plt. subplot (3, 1, 2)
plt. plot (t, X150)
plt. grid ()
plt. ylabel ('X150(t)')
plt. subplot (3, 1, 3)
plt. plot (t, X1500)
plt. grid ()
plt. ylabel ('X1500(t)')
plt. xlabel ('t')

plt. show ()





\end{lstlisting}

\includegraphics{1.png}
\includegraphics{2.png}
\includegraphics{3.png}







\section{Questions}
1.Is x(t) an even or an odd function? Explain why.

	Since x(-t) equal -x(t) then the function id Odd, also a= 0 and b has values

2. Based on your results from Task 1, what do you expect the values of a2, a3, . . . , an to be?
Why?

I expected to be zero because x(t) is an odd function.

3. How does the approximation of the square wave change as the value of N increases? In what
way does the Fourier series struggle to approximate the square wave?

As long as N increases, we will have more accuracy and obvious.

4. What is occurring mathematically in the Fourier series summation as the value of N increases?

The value of x(t) and its figure will be close to the signal square wave.


5. Leave any feedback on the clarity of lab tasks, expectations, and deliverables

Overall, the lab was very clear.






\section{conclusion}
In this lab, we learned about the Fourier Series approximation. We applied Fourier Series. Square wave. Firstly, We. Found the expiration from the  pre-lab and then we display the values of a & b using python and we checked the result. After that, we use the determine and plot the Fourier Series of square wave for different values of  N we found that with the Increase of N, the Fourier Series and the figure will be more accurate and near to the square wave. 


\newpage



\end{document}

%This template was created by Roza Aceska.